\documentclass{article}
\pagestyle{empty}                   % remove page numbers

\let\Item\item
\newcommand\SpecialItem{\renewcommand\item[1][]{\Item[\textbullet~\bfseries##1]}}
\renewcommand\enddescription{\endlist\global\let\item\Item}

\usepackage[colorlinks=true, urlcolor=blue]{hyperref}
\usepackage{xcolor}

% override LaTeX default margins
\addtolength{\topmargin}{-1.0in}
\setlength{\oddsidemargin}{0.5in}
\addtolength{\textwidth}{1.0in}
\addtolength{\marginparsep}{-0.1in}  
\setlength{\parindent}{0mm}         % remove paragraph indents

% shrink text to accommodate a long resume
\setlength{\textheight}{9.5in}
\setlength{\marginparwidth}{1.1in}

\definecolor{lightred}{RGB}{180, 0, 0}

%%%%%%%%%%%%%%%%%%%%%%%%%%%%%%%%%%%%%%%%%%%%%%%%%%%%%%%%%%%%%%%%%%%%%%%%

\begin{document}
\reversemarginpar        % margin notes to the left-hand side for resume

                % address stuff.  The \hspace*{\fill} is
                % necessary to get things laid out
                % correctly.  The sizes fed to \vspace are
                % variable, depending on how much
                % whitespace you want, and how much
				% stuff you're trying to fit on a page....

\fontseries{m}

\begin{tabular}{lr}

  \fontsize{8}{8}
  \selectfont

  & \hspace*{3.05in} \\
  \textcolor{lightred}{{\fontsize{17.28}{17.28} \bf Nandor Tibor Szots}} &
 
  \fontsize{8}{8}
  \selectfont
  \begin{tabular}[b]{l}
    6236 Lisieux Terrace \\
    San Diego, CA 92120 \\
    +1 858 442 1888 \\
    nandor@szots.com
  \end{tabular}

\end{tabular}

\fontsize{12}{13}
\selectfont

\rule{5.7in}{0.02in}
% In 2014 Objective statements are outdated.
%\par
%\vspace{2\baselineskip}
%\marginpar{{\fontsize{12}{13} \bf Objective}}
%To attain a position as a technical director or lead software engineer for a
%cutting-edge technology company.

                % if the \marginpar is at the beginning
                % of the line, it loses.  *sigh*.  This
                % is a typical entry in a resume.   (I'm
                % defining an entry as something
                % important enough to have a marginal note
                % setting it off.)  Give it some space
                % with \vspace, use \par to break
                % between paragraphs, enter your text,
                % and place the \marginpar at the end.

%%%%%%%%%%%%%%%%%%%%%%%%%%%%%%%%%%%%%%%%%%%%%%%%%%%%%%%%%%%%%%%%%%%%%%%%

%\par
%\vspace{2\baselineskip}
%\marginpar{{\fontsize{12}{13} \bf Completed \hspace*{\fill} \linebreak Projects}} 
%\vspace{\baselineskip}
%{\bf Optimizing Compiler} 
%\par
%\vspace{.03in}

%\textit{Frontend}: Designed and implemented an 
%intermediate code generator that takes target source code to an
%intermediate representation via a lexical analyzer,
%LALR(1) parser, and code generator.  Development under Solaris in flex, 
%bison and C. \hfill

%\par
%\vspace{\baselineskip}
%{\bf Linux Kernel Modules} \hspace*{\fill}
%\linebreak
%\textit{Thread Safe System Call}: Added a system call to the Linux 
%kernel, which was thread safe.  Thread safety was implemented with
%both semaphores and wait queues. \hspace*{\fill}
%\linebreak
%\textit{Memory Mapped Device}: Wrote a Linux block device driver for
%a 2 megabyte phantom PC card, which was implemented as two 1 megabyte
%devices, using a memory mapped interface. \hspace*{\fill}
%\linebreak
%\textit{Virtual File System}: Wrote a pure memory virtual file system
%which supported reads, writes, and truncates.

%\textit{http://people.ucsc.edu/\textasciitilde lphillip/TIspecs.pdf}

%%%%%%%%%%%%%%%%%%%%%%%%%%%%%%%%%%%%%%%%%%%%%%%%%%%%%%%%%%%%%%%%%%%%%%%%

\par
\vspace{2\baselineskip}
                % This is a slightly more complicated
                % example.  \vspace gets a `doublespace'
                % (2\baselineskip) between entries, and
                % a  `singlespace' between paragraphs of
                % an entry.
                % (2\baselineskip means twice the value
                % of \baselineskip).  
\marginpar{\textcolor{lightred}{{\fontsize{12}{13}\bf Personal \hspace*{\fill} \linebreak Profile}}}
\par
A self-motivated engineer and a dynamic leader that inspires greatness in his teams.  Possessing extensive
experience in building, leading, mentoring and motivating teams of engineers that consistently deliver projects on time and
reliably in any environment.  Also has a proven track record of designing practical solutions to complex problems - while
ensuring maintainability and reusability.
\par
\vspace{\baselineskip}
\marginpar{\textcolor{lightred}{{\fontsize{12}{13}\bf Work \hspace*{\fill} \linebreak Experience}}}
\vspace{\baselineskip}
{\bf Sony Online Entertainment}
\hfill {\bf January 2013 - Present}
\vspace{.03in}
\par
Projects: EverQuest II, EverQuest Next and EverQuest Landmark
\par
Position: Technical Director
\vspace{.03in}
\par
Worked with other technical personnel on developing new company-wide
technologies.  Helped design and architect key systems for new and existing
games.
\par
\vspace{\baselineskip}
Responsible personnel decisions on EverQuest II.  Interviewed engineers for
positions on both EverQuest II and EverQuest Next.
\par
\vspace{\baselineskip}
Responsible for leading a team of C++ and Web Developers in building the
in-game marketplace and showcase for display and real-time sale of both player and SOE created items - 
\href{https://www.landmarkthegame.com/showcase}{Landmark Showcase}.  This
feature was designed and brought to market in a record 6 months and accounts
for over 50\% of the profits of the game.
\par
\vspace{\baselineskip}
Designed, Architected and Lead the Team in Implementing: 
\vspace{\baselineskip}
\SpecialItem
\begin{itemize}
  \item[]Backend C++ servers which control the searching of and sale of these items.
  \item[]Recommendation system to suggest purchases to users.
  \item[]UI for creating player made items, along with UI Artists  and Producers to make sure all needs were met.
  \item[]Web Based UI through which players can search for and find items to purchase.  This UI was the same both in and out of game to ensure a unified code-base which required minimal effort to keep synchronized between the two UIs.
  \item[]Daily digest e-mail of products which is personalized based on a users interests.
\end{itemize}

\vspace{\baselineskip}
{\bf Sony Online Entertainment}
\hfill {\bf January 2012 - January 2013}
\vspace{.03in}
\par
Project: EverQuest II
\par
Position: Lead Programmer
\vspace{.03in}
\par
Lead a team of programmers in developing new features and
refreshing, enhancing and giving new purpose to previously used features in
the popular EverQuest II MMORPG. 
\par
Responsible for architecting new systems, Full Team
scheduling, project time estimates, and daily tasking of the EverQuest II code team.
\par

\pagebreak

\vspace{\baselineskip}
{\bf Sony Online Entertainment} 
\hfill {\bf January 2006 - January 2012} 
\par
\vspace{.03in}
Project: EverQuest II
\par
Position: Senior Programmer
\vspace{.03in}
\par
Implemented the EverQuest II and EverQuest II: Extended marketplace functionality
including: all back-end commerce and server-side communications.  Transitioned
the original implementation of the marketplace into a highly successful Free-to-Play 
model, while maintaining a non-Free-to-Play service as well.
\par
Responsible for all marketplace improvements, changes and new technology
development.
\par
Responsible for live-game maintenance, including daily investigation of both client
and server crashes by using Linux core dumps and Windows mini-dumps to identify
and resolve issues. 
\par

\vspace{\baselineskip}
Designed and implemented a tool for third-party sites to be able to obtain
information about in-game characters, items and spells.  This system used 
EDB and X-Path to index XML blob data for fast searching of relevant game data.
\par

\vspace{\baselineskip}
Implemented a spam filter for all game messages to combat the growing spam
problems.  The filter has a 99\% accuracy rate and blocks on the order of
500,000 messages per day.
\par

\vspace{\baselineskip}
Updated and maintained game servers and client in C++/STL.  Implemented new 
game systems and features based on player feedback.  Worked with the Test Server 
community to build a stronger relationship between development and players.
\par

\vspace{\baselineskip}
Acted as the game team's point of contact for internationalization across over 4
locales including Russian, French, German, and Japanese.  Developed code which
allowed for faster identification of strings requiring localization and reduced 
redundancy in localizing dynamic game data.
\par

\vspace{\baselineskip}
Created designer tools for fast indexed searching of over 
50 gigabytes of game data.  These tools were rapidly developed using GNU/Linux, 
perl, php, apache and other open source tools.
\par
\vspace{\baselineskip}

\vspace{\baselineskip}
{\bf General Atomics - Lynx Systems} 
\hfill {\bf January 2003 - January 2006} 
\par
Position: Lead Software Engineer
\vspace{.03in}
\par
Wrote embedded C, compiled for vxWorks running on various VME boards.  
\par
Projects included: View Manager Board which displayed radar imagery in near-real-time
using OpenGL; Lynx Ground Control Station: received radar imagery via a
high-speed serial (RS422) data link, decompressed, displayed and forwarded
images to other systems via Ethernet; Utility applications: allowed for
quick and easy loading of code onto radars which consisted of a C back-end
running on the radar and a C\# user interface running on a PC.  
\par

\vspace{\baselineskip}
Trouble shot various hardware and software problems in the field as a radar
integration specialist.  Part of integration team for FireScout, and KingAir.
Software integration lead for Predator B.
\par

\pagebreak


\vspace{\baselineskip}
Software Engineering Lead on SAR/GMTI project.

\vspace{\baselineskip}
{\bf AudioTalk Inc. / HearMe Inc.} 
\hfill {\bf June 1998 - July 2001} 
\par
Position:  Junior Engineer / Operations Engineer
\vspace{.03in}
\par
\textit{Development}: Helped develop various client and server 
MSVC++ applications for HearMe's VoIP chat solution.  Developed 
server side APIs in Perl and C.  The APIs were integrated into 
both WinNT and Linux environments, which handled all the core messaging
in the AudioTalk / HearMe products.
\par
\textit{Operations}: Helped build up, integrate and maintain the AudioTalk and HearMe VoIP 
networks.  This included server setup, installation and maintenance; both 
internally and at customer sites.  Wrote C, Perl and other various CGI 
scripts to automate the network maintenance and ensure stability.

\vspace{\baselineskip}
{\bf University of California, Santa Cruz} 
\hfill {\bf December 1998 - December 2002}
\par
Position:  Student Grader/Lab Assistant/Volunteer Tutor
\par
\vspace{.03in}
Graded for many upper and lower division computer science programming 
classes.  Assisted students with learning the Unix environment, including
make, gcc, gmake, and general Unix commands (ls, ps, vi, etc.)

\vspace{\baselineskip}
{\bf University of California, Santa Cruz} 
\hfill {\bf December 1998 - June 2000}
\par
Position:  Residential Computer Coordinator
\par
\vspace{.03in}
Setup and configured students' computers to work on the school network.
This included explaining the campus e-mail system, and network guidelines 
to both students and faculty.
\par
\vspace{2\baselineskip}
\marginpar{\textcolor{lightred}{{\fontsize{12}{13} \bf Education}}}
{\bf University of California, Santa Cruz} 
\hspace*{\fill} {\bf 1998 - 2002}
\par
\vspace{.03in}
B.S. in Computer Science.
%\textit{Design Classes}: Data Structures \& Algorithms, Computer Architecture,
%Compiler Design, Operating Systems, Comparative Programming Languages, Database Systems;
%\textit{Theory Classes}: Algorithmics, Language Theory, Theory
%of Computation and Complexity, Applied Graph Theory and Algorithms, Software Methodology.

% Shouldn't need to mention the mainstream courses.
\vspace{\baselineskip}
Relevant Skills:  C, C++, Java, C\#, Online Marketplaces, Monetization System
Design, OpenGL, Unix\footnote{Linux: Debian, RedHat, SuSE, Slackware;
Sun: Solaris}, Regular Expressions, vxWorks, Tornado, MS Dev Studio, STL, PL/SQL Dev, vi, gmake, cvs, sh, perl, 
flex, bison, \LaTeX, assembly language\footnote{RISC: SPARC, MIPS, and Motorola MC68xx (MC68HC11A8)
assembly}, HTML, Win9x / NT / 2000 / XP / 7, MS-DOS, SQL,
Databases\footnote{MSDE, Oracle, MySQL, Postgres, EDB, MongoDB}, Perforce, CVS, git, RS-232, RS-422, VME Knowledge.
\vspace{2\baselineskip}

\par

\marginpar{\textcolor{lightred}{{\fontsize{12}{13} \bf References}}}
Available upon request.

\end{document}

